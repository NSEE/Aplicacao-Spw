% ------------------------------------------------------------------------
%  Instituto Mauá de Tecnologia
%  Núcleo de Sistemas Eletrônicos Embarcados - NSEE
%   Rodrigo Romano & Rafael Corsi
% 					 corsiferrao@gmail.com
%  		
%  Projeto CITAR
%	Meta 4
% 	Especificação do produto
% 	Acionamento motor BLDC para rodas de reação
%
%	Junho de 2014
% ------------------------------------------------------------------------
\input{template_nota}
\usepackage{natbib}
%------------------------------------------------------------------------	
\usepackage{tikz}
\graphicspath{ {./figs/} }
% ------------------------------------------------------------------------
\newtcolorbox[auto counter,number within=section]{TBD-BOX}[2][]{%
colback=red!5!white,colframe=darkgray,fonttitle=\bfseries,
title=TBD.~\thetcbcounter: #2,#1}
% ------------------------------------------------------------------------
% http://latexcolor.com/
% cores dos todos
\definecolor{TBC}{rgb}{0.98, 0.91, 0.71}  	%bananamania
\definecolor{Corsi}{rgb}{0.53, 0.66, 0.42}  %
%\definecolor{TBC2}{rgb}{1.0, 0.71, 0.76} 		%lightpink
\definecolor{TBC2}{rgb}{0.86, 0.82, 1.0} %lightmauve
\definecolor{darkgray}{rgb}{0.66, 0.66, 0.66}
\definecolor{aliceblue}{rgb}{0.94, 0.97, 1.0}
% ------------------------------------------------------------------------

\title		{   \LARGE
				Instituto Mauá de Tecnologia \\
				Núcleo de Sistemas Eletrônicos Embarcados - NSEE \\ 
				\vspace{10px}
				\huge
				Especificação do Produto \\
				SpaceWire Data Feed \\
			}
\author		{
				 Rafael Corsi \\ \small 	
				\href{mailto:rafael.corsi@maua.com}{rafael.corsi@maua.com}
			}
%\subject	{Núcleo de Sistemas Eletrônicos Embarcados - NSEE}			
\titlepic	{\includegraphics[width=0.4\textwidth]{maua_logo}}
%\lowertitleback{}

% ------------------------------------------------------------------------
\begin{document}
% ------------------------------------------------------------------------
\maketitle
\tableofcontents
% ------------------------------------------------------------------------
%%---------- Revisão
\begin{versionhistory}
  \vhEntry{0.0.1}{24.6.14}{Corsi}{Criação do documento}
\end{versionhistory}
% ------------------------------------------------------------------------
%%---------- Lista de distribuição
%\section*{Lista de distribuição}
%\todo[inline]{inserir emails}
%\begin{itemize}
%	\item IMT	
%	\begin{itemize}
%		\item Rafael Corsi Ferrão 
%		\item Rodrigo A. Romano 
%		\item Sergio Ribeiro Augusto 
%		\item Vanderlei Cunha Parro
%		\item José Carlos 
%		\item Magda 
%		\item Raphael Ballet
%	\end{itemize}
%	\item INPE
%	\begin{itemize}
%		\item Silvio ?
%	\end{itemize}
%	\item Externo
%	\begin{itemize}
%		\item Saulo
%	\end{itemize}	
%\end{itemize}

% ------------------------------------------------------------------------
% -----------------
\chapter{Introdução}


% -----------------
\section{Finalidade}

O projeto proposta chamado de SpW-Data-Feed é um exemplo de aplicação do protocolo SpaceWire, a ser usado na bancada de experimento do projeto CITAR. 

% -----------------
\section{Escopo Geral}

Definição das especificações gerais do projeto e propostas de implementação.

% -----------------
\section{Referências}


\todo[inline, color=Corsi]{listar documentos de input - IMT e CITAR}

% -----------------
\section{Visão Geral}

O objetivo do projeto é desenvolver um hardware de uso de bancada capaz de ler e escrever em conversores A/D e D/A e em saídas e entradas digitais. O dispositivo será conectado em um nó SpaceWire com as configurações e dados trafegando sob o protocolo RMAP.

O projeto está sendo desenvolvido no âmbito do projeto de Circuitos Integrados Tolerantes a Radiação (CITAR), coordenado pelo pesquisador Saulo Finco do Centro de Tecnologia da Informação Renato Archer (CTI). 


% -----------------
\section{Equipe}

\begin{itemize}
	\item Profissionais do IMT:
	\begin{itemize}
		 \item Rafael Corsi Ferrão (coordenação)
	\end{itemize}
	
	\item Bolsistas:
	\begin{itemize}
		\item IC Dennis Teles (execução)
	\end{itemize}
\end{itemize}


\chapter{Posicionamento}

% ----------------------------------------------------------------
\section{Descrição Geral do Projeto}

Desenvolver um dispositivo embarcado (SpWDataFeed) em uma FPGA capaz de realizar a leitura de conversores A/D, escrever em conversores D/A, acionar e ler entradas digitais. Esse dispositivo será conectado em uma rede SpaceWire com o protocolo RMAP.

O SpWDataFeed será utilizado para demonstrar os benefícios do protocolo SpW em uma aplicação similar ao encontrado em satélites.

% ----------------
\section{Representação Gráfica do Produto}

A arquitetura proposta para a implementação do SpWDataFeed é ilustrado na Fig. \ref{fig:arch}, onde o acesso ao dispositivo é feito via comandos RMAP. Uma interface seria RS232 é utilizada para configuração e debug do dispositivo. 

O módulo implementa leitura de conversores A/D (SpI, I2C, paralelo (\textbf{TBD})) através do A/D Módulo, escrita em conversores D/A (\textbf{TBD}) pelo D/A módulo e acesso direto a I/Os digitais via os módulos Din e Dout.

\begin{figure}[ht!]
	\centering
	\includegraphics[width=0.8\linewidth]{./figs/arch}
	\caption{Diagrama de blocos}
	\label{fig:arch}
\end{figure}

A arquitetura é concebida de forma que o dispositivo possa ser utilizado para controle e acionamento de diversos sistemas aerospaciais, tais como: abertura de painel solar, controle de experimento.

\chapter{Identificação dos Componentes Envolvidos e Responsáveis}

\section{SpaceWire Codec}

IP core que implementa o protocolo SpaceWire, pretende-se utilizar o codec desenvolvido no projeto CITAR mas existem outras opções, tais como :

\begin{itemize}
	\item SpaceWire NSEE
	\item SpaceWire Light
	\item SpaceWire CEA
\end{itemize}

\section{RMAP}

O protocolo RMAP será a forma de acesso e controle da aplicação, via comandos de escrita, será possível configurar os modos de operação assim como escrever nos conversores D/A e saídas digitais. Comandos de leitura serão utilizados para acesso assíncrono ao dados dos conversores A/D e portas digitais de entrada. Comandos de escrita da aplicação para o nó remoto, será utilizado para envio periódico das informações desses sensores.

Estudar viabilidade de uso do código já implementado para o Simucam. O comando de Read-Write não é implementado nesse core.

\section{OTC - On Board Time Controller}

O OTC será o responsável pela gestão dos sinais de sincronismo internos assim como o fornecedor de timestamp para o envio junto com os dados. Esse bloco é sincronizado com o restante dos nós via comandos de timecode enviados no canal SpW.

\section{Controller}

Core responsável por implementar a máquina de estados que controla a aplicação. Estudar a possibilidade de embarcar um uP.

\section{Reg. Banco}

Banco de registros compartilhado entre os módulos, para armazenamento de dados e configurações.

\section{Módulo A/D}

Módulo responsável pelo controle dos conversores A/D e escrita dos dados no Reg. Banco.

\section{Módulo D/A}

Módulo responsável pelo controle dos conversores D/A com os parâmetros configurados no Reg. Banco.

\section{Módulo Din}

Módulo responsável pelo controle das entradas digitais e escrita dos dados no Reg. Banco.

\section{Módulo Dout}	

Módulo responsável pelo controle das saídas digitais com os parâmetros configurados no Reg. Banco.

\section{Comandos}


\subsection{Definir zona de memória}

% ----------------
\chapter{Recursos do Produto}

\section{Fases do projeto}

\begin{enumerate}
	\item Especificação detalhada
	\item Implementação
	\item Teste 
\end{enumerate}

\section{Ferramentas}

Para implementação inicial, pretende-se utilizar a placa de desenvolvimento GR-PCI-XC5V da pender que possui uma FPGA Xilinx Virtex 5. O desenvolvimento será feito no software ISE com simulações tanto no ISIM quanto no ModelSim.

Para testes a nível SpW, utilizar o USB Brick da StarDundee.

\section{Gestão}



% ----------------
	\apptocmd{\thebibliography}{\footnotesize}{}{}
	\bibliographystyle{unsrt}
	\addcontentsline{toc}{section}{Referências Bibliográficas}
	\bibliography{bibliografia}

%-----------------------------------------------------
\end{document}
% ------------------------------------------------------------------------

% ------------------------------------------------------------------------
% COLAS
% ------------------------------------------------------------------------

% ------------------------------------------------------------------------
%\begin{figure}[ht!]  
%	\centering
%		\includegraphics[width=0.6 \columnwidth,angle=0]{figs/svpwm.pdf}
%	\caption{Acionamento SVPWM}
%	\label{fig:svpwm}
%\end{figure}
% ------------------------------------------------------------------------

% ------------------------------------------------------------------------
%\begin{tcolorbox}[title=Objetivo central]
%	Desenvolver um dispositivo embarcado em uma FPGA capaz de realizar tanto o acionamento como o controle de motores de corrente contínua sem escovas (BLDC)
%\end{tcolorbox}
% ------------------------------------------------------------------------
%\begin{TBD-BOX}[label={myautocounter}]{Metodologia}
%	\begin{itemize}
%		\item Verificar especificação com CITAR
%		\item TestBenchs
%	\end{itemize}
	%\end{TBD-BOX}